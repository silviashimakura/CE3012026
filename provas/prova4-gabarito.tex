% Options for packages loaded elsewhere
\PassOptionsToPackage{unicode}{hyperref}
\PassOptionsToPackage{hyphens}{url}
%
\documentclass[
  12pt,
]{article}
\usepackage{amsmath,amssymb}
\usepackage{iftex}
\ifPDFTeX
  \usepackage[T1]{fontenc}
  \usepackage[utf8]{inputenc}
  \usepackage{textcomp} % provide euro and other symbols
\else % if luatex or xetex
  \usepackage{unicode-math} % this also loads fontspec
  \defaultfontfeatures{Scale=MatchLowercase}
  \defaultfontfeatures[\rmfamily]{Ligatures=TeX,Scale=1}
\fi
\usepackage{lmodern}
\ifPDFTeX\else
  % xetex/luatex font selection
\fi
% Use upquote if available, for straight quotes in verbatim environments
\IfFileExists{upquote.sty}{\usepackage{upquote}}{}
\IfFileExists{microtype.sty}{% use microtype if available
  \usepackage[]{microtype}
  \UseMicrotypeSet[protrusion]{basicmath} % disable protrusion for tt fonts
}{}
\makeatletter
\@ifundefined{KOMAClassName}{% if non-KOMA class
  \IfFileExists{parskip.sty}{%
    \usepackage{parskip}
  }{% else
    \setlength{\parindent}{0pt}
    \setlength{\parskip}{6pt plus 2pt minus 1pt}}
}{% if KOMA class
  \KOMAoptions{parskip=half}}
\makeatother
\usepackage{xcolor}
\usepackage[left=2cm,right=2cm,top=0.5cm,bottom=1.5cm]{geometry}
\usepackage{graphicx}
\makeatletter
\newsavebox\pandoc@box
\newcommand*\pandocbounded[1]{% scales image to fit in text height/width
  \sbox\pandoc@box{#1}%
  \Gscale@div\@tempa{\textheight}{\dimexpr\ht\pandoc@box+\dp\pandoc@box\relax}%
  \Gscale@div\@tempb{\linewidth}{\wd\pandoc@box}%
  \ifdim\@tempb\p@<\@tempa\p@\let\@tempa\@tempb\fi% select the smaller of both
  \ifdim\@tempa\p@<\p@\scalebox{\@tempa}{\usebox\pandoc@box}%
  \else\usebox{\pandoc@box}%
  \fi%
}
% Set default figure placement to htbp
\def\fps@figure{htbp}
\makeatother
\setlength{\emergencystretch}{3em} % prevent overfull lines
\providecommand{\tightlist}{%
  \setlength{\itemsep}{0pt}\setlength{\parskip}{0pt}}
\setcounter{secnumdepth}{-\maxdimen} % remove section numbering
\usepackage[utf8]{inputenc}
\usepackage[T1]{fontenc}
\usepackage[portuguese]{babel}
\usepackage{hyphenat}
\usepackage{float}
\usepackage{placeins}
\usepackage{mathtools}
\usepackage{amsmath}
\usepackage{natbib}
\usepackage{arydshln}
\usepackage{multirow}
\usepackage{booktabs}
\usepackage{caption}
\usepackage{fancyhdr}
\usepackage{multirow}
\usepackage{comment}
\newcommand{\R}{{\normalfont\textsf{R}}{}}
\includecomment{solucao}
\usepackage{bookmark}
\IfFileExists{xurl.sty}{\usepackage{xurl}}{} % add URL line breaks if available
\urlstyle{same}
\hypersetup{
  pdftitle={CE301 - Estatística Básica - Prova 4},
  hidelinks,
  pdfcreator={LaTeX via pandoc}}

\title{CE301 - Estatística Básica - Prova 4}
\author{}
\date{\vspace{-2.5em}1o. Semestre 2025}

\begin{document}
\maketitle

\noindent \textbf{Nome:} \underline{\hspace{16cm}}\\

\textbf{Data:} \underline{\hspace{1cm}/\hspace{1cm}/\hspace{1cm}}
\hspace{0.5cm} \textbf{GRR:} \underline{\hspace{3cm}} \hspace{0.5cm}
\textbf{Assinatura:} \underline{\hspace{5cm}}

\vspace{0.5cm}

\begin{enumerate}
\def\labelenumi{\arabic{enumi})}
\tightlist
\item
  Uma refinaria recebe navios transportando petróleo bruto.
  Historicamente, a quantidade de carga por navio segue distribuição
  normal com média \(\mu=30000\) toneladas, mas o desvio-padrão
  populacional \(\sigma\) é desconhecido.
\end{enumerate}

Você coleta uma amostra aleatória simples de \(n=25\) navios, e obtém
uma média amostral de \(\bar{x}=28000\) toneladas e um desvio-padrão
amostral de \(s=4200\) toneladas.

\begin{enumerate}
\def\labelenumi{\alph{enumi})}
\item
  Liste e justifique brevemente as condições necessárias para aplicar
  testes de média e variância nesses dados (independência, normalidade,
  tamanho da amostra, uso de \(t\) ou qui-quadrado). (0,5 ponto)
\item
  Construa o intervalo de confiança de 95\% para \(\mu\). Qual
  distribuição foi usada? Interprete o resultado no contexto da
  refinaria. (0,5 ponto)
\item
  A refinaria suspeita que a média atual é menor que a histórica de
  30000 t.

  \begin{itemize}
  \item[i)] Formule a hipótese nula e a alternativa? (0,5 ponto)
  \item[ii)] Calcule o valor da estatística de teste, a região crítica e o p-valor? (2,0 pontos)
  \item[iii)] Qual é a conclusão do teste com nível de significância de 5\%? Interpretação prática. (0,5 ponto)
  \end{itemize}
\item
  Obtenha um intervalo de confiança de 95\% para \(\sigma^2\). Quais
  parâmetros foram usados (graus de liberdade, distribuição)?
  Interprete. (0,5 ponto)
\item
  Calcule o tamanho de amostra necessário para que a margem de erro
  máxima para estimar \(\mu\) seja 500 toneladas, com 99\% de confiança.
  Use como estimativa substituta \(s = 4200\) t. Interprete esse
  resultado. (0,5 ponto)
\end{enumerate}

\begin{solucao}
\hrulefill

\textbf{Solução}
\begin{itemize}
\item[a)] Suposições:
\begin{itemize}
\item Amostra aleatória $\rightarrow$ independência
\item População aproximadamente normal (ou $n$ moderado) $\rightarrow$ uso de **t** e **qui-quadrado**
\end{itemize}

\item[b)] IC 95\% para $\mu$: aproximadamente [\ensuremath{2.6266\times 10^{4}}; \ensuremath{2.9734\times 10^{4}}] toneladas. 

\textbf{Interpretação}: Há 95\% de confiança de que este intervalo cobre a verdadeira média.

\item[c)] Teste unilateral $H_0: \mu=30000$ vs $H_1:\mu<30000$

\begin{itemize}
\item Estatística de teste: $t_{calc}=$ -2.3809524
\item $RC=\{t_{calc}\leq$ -2.064$\}$
\item p-valor=0.0127759 $< 0,05 \rightarrow $ Rejeitamos $H_0$.  
\end{itemize}

\textbf{Conclusão}: Há evidência suficiente de que a média atual é menor que 30000 t.


\item[d)] IC 95\% para $\sigma^2$: [\ensuremath{1.0754984\times 10^{7}}; \ensuremath{3.4138769\times 10^{7}}] ton$^2$

\textbf{Interpretação}: Há 95\% de confiança de que este intervalo cobre a verdadeira variância das cargas.

\item[e)] \textbf{Tamanho amostral} p/ $ME = 500$ (99\% conf.): 469 navios. 

\textbf{Interpretação}: Para uma estimativa tão precisa, seria necessário uma amostra bem maior.


\end{itemize}

\end{solucao}

\hrulefill

\begin{enumerate}
\def\labelenumi{\arabic{enumi})}
\setcounter{enumi}{1}
\tightlist
\item
  Uma fábrica visa que no máximo 2\% das peças produzidas sejam
  defeituosas (\(p\leq 0,02\)). Em um lote recente, foi sorteada uma
  amostra aleatória simples de \(n = 500\) peças e observou-se 12 peças
  defeituosas.
\end{enumerate}

\begin{enumerate}
\def\labelenumi{\alph{enumi})}
\item
  Identifique o parâmetro de interesse. Informe a estimativa pontual.
  (0,5 ponto)
\item
  Calcule o erro padrão da proporção estimada. Verifique as condições
  para aplicar a aproximação normal (\(np_0\geq 5\) e
  \(n(1-p_0)\geq 5\)). (0,5 ponto)
\item
  Monte o intervalo de confiança otimista de 95\% para a proporção
  populacional de peças defeituosas. Interprete o resultado em relação
  ao padrão da fábrica. (1,0 ponto)
\item
  Estabeleça \(H_0\) e \(H_1\) para testar se \(p>0,02\). Calcule a
  estatística de teste e a região crítica. Decisão a um nível de
  significância \(\alpha=5\)\%. Há evidência de que a taxa de defeitos
  excede o limite desejado? (2,0 pontos)
\item
  Especifique o p-valor calculado e explique seu significado no contexto
  decisional. (0,5 ponto)
\item
  Quer-se reduzir a margem de erro para metade da atual (considerando o
  intervalo de confiança prévio) com 95\% de confiança. Utilize valor
  otimista \(p=0,02\) (maximiza precisão). Calcule o novo \(n\).
  Explique como interpretar esse tamanho conforme investimento em
  controle de qualidade. (0,5 ponto)
\end{enumerate}

\begin{solucao}
\hrulefill

\textbf{Solução}

\begin{enumerate}
\item[a)] Parâmetro e estimativa
\begin{itemize}
\item Parâmetro: proporção populacional $p$
\item Estimativa pontual: $\hat{p} = 0.024$
\end{itemize}

\item[b)] Erro padrão e condições
\begin{align*}
se &= 0.0068446\\
np_0 &= 10\geq 5\\
nq_0 &= 490\geq 5\\
& \mbox{condições atendidas}
\end{align*}

\item[c)] IC 95\% para proporção
\begin{align*}
\mbox{Intervalo= } & [0.0106;0.0374]\\
\mbox{Interpretação: }& \mbox{A verdadeira taxa de defeitos é coberta pelo IC com uma confiança de 95}\%. 
\end{align*}

\item[d)] Teste unilateral $H_0:p=0,02$ vs $H_1:p>0,02$
\begin{align*}
\mbox{Estatística de teste: }& z=0.58\\
\mbox{p-valor: }& 0.2795  > 0.05 \rightarrow \mbox{não rejeita} H_0\\
\mbox{Conclusão: }& \mbox{ sem evidência de que a proporção excede 2}\%.
\end{align*}

\item[e)] Valor-p = 0.2795. 
\begin{align*}
\mbox{Interpretação: }& \mbox{ Sob } H_0, \mbox{há } 27.9\% \mbox{de chance de observar } \hat p \mbox{ tão alto ou maior;} \\
&\mbox{Não é suficiente para rejeitar.}
\end{align*}

\item[f)] Tamanho amostral p/ metade do erro (95\%)

Resultado: 1674 peças. 

Interpretação: aumentaria a amostra para reduzir o erro pela metade.

\end{enumerate}
\end{solucao}

\hrulefill

\end{document}
