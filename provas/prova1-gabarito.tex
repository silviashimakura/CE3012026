% Options for packages loaded elsewhere
\PassOptionsToPackage{unicode}{hyperref}
\PassOptionsToPackage{hyphens}{url}
%
\documentclass[
]{article}
\usepackage{amsmath,amssymb}
\usepackage{iftex}
\ifPDFTeX
  \usepackage[T1]{fontenc}
  \usepackage[utf8]{inputenc}
  \usepackage{textcomp} % provide euro and other symbols
\else % if luatex or xetex
  \usepackage{unicode-math} % this also loads fontspec
  \defaultfontfeatures{Scale=MatchLowercase}
  \defaultfontfeatures[\rmfamily]{Ligatures=TeX,Scale=1}
\fi
\usepackage{lmodern}
\ifPDFTeX\else
  % xetex/luatex font selection
\fi
% Use upquote if available, for straight quotes in verbatim environments
\IfFileExists{upquote.sty}{\usepackage{upquote}}{}
\IfFileExists{microtype.sty}{% use microtype if available
  \usepackage[]{microtype}
  \UseMicrotypeSet[protrusion]{basicmath} % disable protrusion for tt fonts
}{}
\makeatletter
\@ifundefined{KOMAClassName}{% if non-KOMA class
  \IfFileExists{parskip.sty}{%
    \usepackage{parskip}
  }{% else
    \setlength{\parindent}{0pt}
    \setlength{\parskip}{6pt plus 2pt minus 1pt}}
}{% if KOMA class
  \KOMAoptions{parskip=half}}
\makeatother
\usepackage{xcolor}
\usepackage[left=2cm,right=2cm,top=0.5cm,bottom=2cm]{geometry}
\usepackage{color}
\usepackage{fancyvrb}
\newcommand{\VerbBar}{|}
\newcommand{\VERB}{\Verb[commandchars=\\\{\}]}
\DefineVerbatimEnvironment{Highlighting}{Verbatim}{commandchars=\\\{\}}
% Add ',fontsize=\small' for more characters per line
\usepackage{framed}
\definecolor{shadecolor}{RGB}{248,248,248}
\newenvironment{Shaded}{\begin{snugshade}}{\end{snugshade}}
\newcommand{\AlertTok}[1]{\textcolor[rgb]{0.94,0.16,0.16}{#1}}
\newcommand{\AnnotationTok}[1]{\textcolor[rgb]{0.56,0.35,0.01}{\textbf{\textit{#1}}}}
\newcommand{\AttributeTok}[1]{\textcolor[rgb]{0.13,0.29,0.53}{#1}}
\newcommand{\BaseNTok}[1]{\textcolor[rgb]{0.00,0.00,0.81}{#1}}
\newcommand{\BuiltInTok}[1]{#1}
\newcommand{\CharTok}[1]{\textcolor[rgb]{0.31,0.60,0.02}{#1}}
\newcommand{\CommentTok}[1]{\textcolor[rgb]{0.56,0.35,0.01}{\textit{#1}}}
\newcommand{\CommentVarTok}[1]{\textcolor[rgb]{0.56,0.35,0.01}{\textbf{\textit{#1}}}}
\newcommand{\ConstantTok}[1]{\textcolor[rgb]{0.56,0.35,0.01}{#1}}
\newcommand{\ControlFlowTok}[1]{\textcolor[rgb]{0.13,0.29,0.53}{\textbf{#1}}}
\newcommand{\DataTypeTok}[1]{\textcolor[rgb]{0.13,0.29,0.53}{#1}}
\newcommand{\DecValTok}[1]{\textcolor[rgb]{0.00,0.00,0.81}{#1}}
\newcommand{\DocumentationTok}[1]{\textcolor[rgb]{0.56,0.35,0.01}{\textbf{\textit{#1}}}}
\newcommand{\ErrorTok}[1]{\textcolor[rgb]{0.64,0.00,0.00}{\textbf{#1}}}
\newcommand{\ExtensionTok}[1]{#1}
\newcommand{\FloatTok}[1]{\textcolor[rgb]{0.00,0.00,0.81}{#1}}
\newcommand{\FunctionTok}[1]{\textcolor[rgb]{0.13,0.29,0.53}{\textbf{#1}}}
\newcommand{\ImportTok}[1]{#1}
\newcommand{\InformationTok}[1]{\textcolor[rgb]{0.56,0.35,0.01}{\textbf{\textit{#1}}}}
\newcommand{\KeywordTok}[1]{\textcolor[rgb]{0.13,0.29,0.53}{\textbf{#1}}}
\newcommand{\NormalTok}[1]{#1}
\newcommand{\OperatorTok}[1]{\textcolor[rgb]{0.81,0.36,0.00}{\textbf{#1}}}
\newcommand{\OtherTok}[1]{\textcolor[rgb]{0.56,0.35,0.01}{#1}}
\newcommand{\PreprocessorTok}[1]{\textcolor[rgb]{0.56,0.35,0.01}{\textit{#1}}}
\newcommand{\RegionMarkerTok}[1]{#1}
\newcommand{\SpecialCharTok}[1]{\textcolor[rgb]{0.81,0.36,0.00}{\textbf{#1}}}
\newcommand{\SpecialStringTok}[1]{\textcolor[rgb]{0.31,0.60,0.02}{#1}}
\newcommand{\StringTok}[1]{\textcolor[rgb]{0.31,0.60,0.02}{#1}}
\newcommand{\VariableTok}[1]{\textcolor[rgb]{0.00,0.00,0.00}{#1}}
\newcommand{\VerbatimStringTok}[1]{\textcolor[rgb]{0.31,0.60,0.02}{#1}}
\newcommand{\WarningTok}[1]{\textcolor[rgb]{0.56,0.35,0.01}{\textbf{\textit{#1}}}}
\usepackage{longtable,booktabs,array}
\usepackage{calc} % for calculating minipage widths
% Correct order of tables after \paragraph or \subparagraph
\usepackage{etoolbox}
\makeatletter
\patchcmd\longtable{\par}{\if@noskipsec\mbox{}\fi\par}{}{}
\makeatother
% Allow footnotes in longtable head/foot
\IfFileExists{footnotehyper.sty}{\usepackage{footnotehyper}}{\usepackage{footnote}}
\makesavenoteenv{longtable}
\usepackage{graphicx}
\makeatletter
\def\maxwidth{\ifdim\Gin@nat@width>\linewidth\linewidth\else\Gin@nat@width\fi}
\def\maxheight{\ifdim\Gin@nat@height>\textheight\textheight\else\Gin@nat@height\fi}
\makeatother
% Scale images if necessary, so that they will not overflow the page
% margins by default, and it is still possible to overwrite the defaults
% using explicit options in \includegraphics[width, height, ...]{}
\setkeys{Gin}{width=\maxwidth,height=\maxheight,keepaspectratio}
% Set default figure placement to htbp
\makeatletter
\def\fps@figure{htbp}
\makeatother
\setlength{\emergencystretch}{3em} % prevent overfull lines
\providecommand{\tightlist}{%
  \setlength{\itemsep}{0pt}\setlength{\parskip}{0pt}}
\setcounter{secnumdepth}{-\maxdimen} % remove section numbering
\usepackage[utf8]{inputenc}
\usepackage[T1]{fontenc}
\usepackage[portuguese]{babel}
\usepackage{hyphenat}
\usepackage{float}
\usepackage{placeins}
\usepackage{mathtools}
\usepackage{amsmath}
\usepackage{natbib}
\usepackage{arydshln}
\usepackage{multirow}
\usepackage{booktabs}
\usepackage{caption}
\usepackage{fancyhdr}
\usepackage{multirow}
\ifLuaTeX
  \usepackage{selnolig}  % disable illegal ligatures
\fi
\usepackage{bookmark}
\IfFileExists{xurl.sty}{\usepackage{xurl}}{} % add URL line breaks if available
\urlstyle{same}
\hypersetup{
  pdftitle={CE301 - Estatística Básica - Prova 1},
  hidelinks,
  pdfcreator={LaTeX via pandoc}}

\title{CE301 - Estatística Básica - Prova 1}
\author{}
\date{\vspace{-2.5em}1o. Semestre 2025}

\begin{document}
\maketitle

\noindent \textbf{Nome:} \underline{\hspace{16cm}}\\

\textbf{Data:} \underline{\hspace{1cm}/\hspace{1cm}/\hspace{1cm}}
\hspace{0.5cm} \textbf{GRR:} \underline{\hspace{3cm}} \hspace{0.5cm}
\textbf{Assinatura:} \underline{\hspace{5cm}}

\vspace{1cm}

Um questionário foi aplicado a uma amostra de 10 alunos de um curso de
MBA de uma universidade americana, fornecendo as seguintes informações:
\textbf{love} (1: sem relacionamento estável, 2: relacionamento estável,
3: sentimento profundo de pertencimento e cuidado), \textbf{sex} (1:
atividade sexual satisfatória, 0: não), \textbf{work} (escala de 5
pontos em que 1: sem trabalho, 3: trabalho ok, 5: amo meu trabalho),
\textbf{money} (renda familiar em milhares de dólares) e \textbf{happy}
(felicidade numa escala de 10 pontos em que 10 é o mais feliz). Os dados
da amostra estão descritos na Tabela 1.

\begin{longtable}[]{@{}rrrrr@{}}
\caption{Dados da amostra}\tabularnewline
\toprule\noalign{}
happy & money & sex & love & work \\
\midrule\noalign{}
\endfirsthead
\toprule\noalign{}
happy & money & sex & love & work \\
\midrule\noalign{}
\endhead
\bottomrule\noalign{}
\endlastfoot
7 & 90 & 1 & 2 & 2 \\
8 & 45 & 1 & 3 & 4 \\
2 & 0 & 0 & 2 & 2 \\
8 & 35 & 1 & 3 & 3 \\
8 & 62 & 0 & 3 & 4 \\
6 & 45 & 0 & 2 & 3 \\
5 & 70 & 0 & 2 & 3 \\
4 & 88 & 1 & 1 & 2 \\
7 & 40 & 0 & 2 & 3 \\
5 & 56 & 1 & 2 & 3 \\
\end{longtable}

Com base nos dados, responda as questões de 1 a 11.

Nas respostas use pelo menos 2 casas decimais.

Nos gráficos atente-se para a legenda e escalas.

\begin{center}
\underline{\hspace{10cm}}
\end{center}

\begin{enumerate}
\def\labelenumi{\arabic{enumi})}
\tightlist
\item
  Quais são os tipos de variáveis coletadas? Classifique-as em
  qualitativa nominal, qualitativa ordinal, quantitativa discreta e
  quantitativa contínua. (0,5 ponto)
\end{enumerate}

\begin{itemize}
\item
  happy: quantitativa contínua
\item
  money: quantitativa contínua
\item
  sex: qualitativa nominal
\item
  love: qualitativa ordinal
\item
  work: quantitativa continua/qualitativa ordinal
\end{itemize}

\begin{enumerate}
\def\labelenumi{\arabic{enumi})}
\setcounter{enumi}{1}
\tightlist
\item
  Considere que exista um cadastro de alunos alocados em 5 turmas.
  Primeiramente foi selecionada uma turma por meio de um sorteio em que
  todas as turmas tinham a mesma probabilidade de fazer parte da
  amostra. Após o sorteio da turma, uma amostra aleatória de alunos da
  turma sorteada foi selecionada em que todos os elementos tinham a
  mesma probabilidade de serem selecionados. Qual o tipo de amostragem
  utilizada? Este plano de amostragem corresponde a um método
  probabilístico ou não probabilístico? Justifique sua resposta. (0,5
  ponto)
\end{enumerate}

Amostragem por conglomerados: método probabilístico.

\begin{enumerate}
\def\labelenumi{\arabic{enumi})}
\setcounter{enumi}{2}
\tightlist
\item
  Monte uma tabela de frequências para a variável \textbf{work}. Use
  frequências absolutas e relativas. Qual seria o gráfico mais adequado
  para representar esta tabela? (0,5 ponto)
\end{enumerate}

\begin{longtable}[]{@{}lrr@{}}
\toprule\noalign{}
work & f & fr \\
\midrule\noalign{}
\endhead
\bottomrule\noalign{}
\endlastfoot
2 & 3 & 0.3 \\
3 & 5 & 0.5 \\
4 & 2 & 0.2 \\
\end{longtable}

\begin{enumerate}
\def\labelenumi{\arabic{enumi})}
\setcounter{enumi}{3}
\tightlist
\item
  Monte uma tabela de frequências para a variável \textbf{money}. Use
  faixas de tamanho 20, partindo de 0 até 100. Qual é a faixa modal?
  (0,5 ponto)
\end{enumerate}

\includegraphics{prova1-gabarito_files/figure-latex/unnamed-chunk-5-1.pdf}

\begin{enumerate}
\def\labelenumi{\arabic{enumi})}
\setcounter{enumi}{4}
\tightlist
\item
  Obtenha média e desvio-padrão das variáveis \textbf{money} e
  \textbf{happy}. (1 ponto)
\end{enumerate}

\begin{Shaded}
\begin{Highlighting}[]
\FunctionTok{mean}\NormalTok{(dados}\SpecialCharTok{$}\NormalTok{money)}
\end{Highlighting}
\end{Shaded}

\begin{verbatim}
## [1] 53.1
\end{verbatim}

\begin{Shaded}
\begin{Highlighting}[]
\FunctionTok{mean}\NormalTok{(dados}\SpecialCharTok{$}\NormalTok{happy)}
\end{Highlighting}
\end{Shaded}

\begin{verbatim}
## [1] 6
\end{verbatim}

\begin{enumerate}
\def\labelenumi{\arabic{enumi})}
\setcounter{enumi}{5}
\tightlist
\item
  As variáveis \textbf{happy} e \textbf{money} estão em diferentes
  escalas, qual delas apresenta maior variabilidade? Utiliza uma medida
  de comparação adequada. (1 ponto)
\end{enumerate}

\begin{Shaded}
\begin{Highlighting}[]
\FunctionTok{sd}\NormalTok{(dados}\SpecialCharTok{$}\NormalTok{happy)}\SpecialCharTok{/}\FunctionTok{mean}\NormalTok{(dados}\SpecialCharTok{$}\NormalTok{happy)}
\end{Highlighting}
\end{Shaded}

\begin{verbatim}
## [1] 0.3333333
\end{verbatim}

\begin{Shaded}
\begin{Highlighting}[]
\FunctionTok{sd}\NormalTok{(dados}\SpecialCharTok{$}\NormalTok{money)}\SpecialCharTok{/}\FunctionTok{mean}\NormalTok{(dados}\SpecialCharTok{$}\NormalTok{money)}
\end{Highlighting}
\end{Shaded}

\begin{verbatim}
## [1] 0.5023109
\end{verbatim}

\begin{enumerate}
\def\labelenumi{\arabic{enumi})}
\setcounter{enumi}{6}
\item
  Com base na tabela do item (4), esboce o histograma da variável
  \textbf{money}. O que você conclui a respeito da simetria? (1 ponto)
\item
  Obtenha as quantidades necessárias e esboce o box-plot da variável
  \textbf{happy}. Coloque nos eixos os valores utilizados para o esboço.
  O que você conclui a respeito da simetria e da presença de valores
  atípicos? (1 ponto)
\end{enumerate}

\begin{Shaded}
\begin{Highlighting}[]
\FunctionTok{boxplot}\NormalTok{(dados}\SpecialCharTok{$}\NormalTok{happy)}
\end{Highlighting}
\end{Shaded}

\includegraphics{prova1-gabarito_files/figure-latex/unnamed-chunk-8-1.pdf}

\begin{Shaded}
\begin{Highlighting}[]
\FunctionTok{summary}\NormalTok{(dados}\SpecialCharTok{$}\NormalTok{happy)}
\end{Highlighting}
\end{Shaded}

\begin{verbatim}
##    Min. 1st Qu.  Median    Mean 3rd Qu.    Max. 
##    2.00    5.00    6.50    6.00    7.75    8.00
\end{verbatim}

\begin{enumerate}
\def\labelenumi{\arabic{enumi})}
\setcounter{enumi}{8}
\tightlist
\item
  Monte uma tabela de dupla entrada usando frequências absolutas para
  \textbf{love} e \textbf{sex}. O que você conclui? (1 ponto)
\end{enumerate}

\begin{Shaded}
\begin{Highlighting}[]
\FunctionTok{table}\NormalTok{(dados}\SpecialCharTok{$}\NormalTok{love,dados}\SpecialCharTok{$}\NormalTok{sex)}
\end{Highlighting}
\end{Shaded}

\begin{verbatim}
##    
##     0 1
##   1 0 1
##   2 4 2
##   3 1 2
\end{verbatim}

\begin{enumerate}
\def\labelenumi{\arabic{enumi})}
\setcounter{enumi}{9}
\tightlist
\item
  Avalie os gráficos abaixo. O que você conclui a respeito da relação
  entre as variáveis \textbf{happy} e \textbf{money}? (1 ponto)
\end{enumerate}

\includegraphics{prova1-gabarito_files/figure-latex/unnamed-chunk-10-1.pdf}

\begin{enumerate}
\def\labelenumi{\arabic{enumi})}
\setcounter{enumi}{10}
\tightlist
\item
  Obtenha uma medida de associação entre \textbf{sex} e \textbf{love}. O
  que você conclui? (1 ponto)
\end{enumerate}

\begin{Shaded}
\begin{Highlighting}[]
\FunctionTok{chisq.test}\NormalTok{(}\FunctionTok{table}\NormalTok{(dados}\SpecialCharTok{$}\NormalTok{love,dados}\SpecialCharTok{$}\NormalTok{sex))}
\end{Highlighting}
\end{Shaded}

\begin{verbatim}
## Warning in chisq.test(table(dados$love, dados$sex)): Aproximação do
## qui-quadrado pode estar incorreta
\end{verbatim}

\begin{verbatim}
## 
##  Pearson's Chi-squared test
## 
## data:  table(dados$love, dados$sex)
## X-squared = 2, df = 2, p-value = 0.3679
\end{verbatim}

\begin{enumerate}
\def\labelenumi{\arabic{enumi})}
\setcounter{enumi}{11}
\tightlist
\item
  Responda de forma sucinta: (1 ponto)
\end{enumerate}

\begin{enumerate}
\def\labelenumi{\alph{enumi})}
\tightlist
\item
  Qual a diferença entre amostragem casual simples e amostragem
  sistemática?
\item
  Por que é melhor evitar o gráfico de setores?
\item
  Uma medida de tendência central é suficiente para representar uma
  variável? Explique.
\item
  O que é um ponto atípico ou outlier?
\item
  O que o coeficiente de correlação expressa? Comente suas
  características.
\end{enumerate}

\begin{center}
\underline{\hspace{10cm}}
\end{center}

\begin{center}
$\bar y= \frac{\sum_{i=1}^n y_i}{n}$ \hspace{2cm} $\bar y= \frac{\sum_{i=1}^k f_i \cdot y_i}{\sum_{i=1}^k f_i}$ \hspace{2cm} $A=max(y)-min(y)$

$DAM_{média}=\frac{1}{n}\sum_{i=1}^n|y_i-\bar y|$ \hspace{2cm} $DAM_{mediana}=\frac{1}{n}\sum_{i=1}^n|y_i-md|$

$s^2=Var(y)=\frac{\sum_{i=1}^n(y_i-\bar y)^2}{n-1}=\frac{1}{n-1}\left(\sum_{i=1}^ny_i^2-\frac{\left(\sum_{i=1}y_i\right)^2}{n}\right)$ \hspace{2cm} $s=\sqrt{s^2}$

$CV=100 \cdot \frac{s}{\bar y}$ \hspace{1cm} $z=\frac{y_i-\bar y}{s}$ \hspace{1cm} $H=-\sum_{i=1}^Sf_i \cdot ln(fi)$ \hspace{1cm} $Q=\sum_{i=1}^r\sum_{j=1}^s \frac{(o_{ij}-e_{ij})^2}{e_{ij}}$

$Cov(y_1,y_2)=\frac{\sum_{i=1}^n(y_{1i}-\bar y_1)(y_{2i}-\bar y_2)}{n-1}$ \hspace{2cm} $r=\frac{\sum_{i=1}^n (y_{1i}-\bar y_1)(y_{2i}-\bar y_2)}{\sqrt{\sum_{i=1}^n(y1i-\bar y_1)^2} \cdot \sqrt{\sum_{i=1}^n (y_{2i}-\bar y_2)^2}}=\frac{Cov(y_1,y_2)}{\sqrt{Var(y_1)} \cdot \sqrt{Var(y_2)}}$

\underline{\hspace{10cm}}

\end{center}

\end{document}
